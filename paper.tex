\documentclass[conference]{resources/IEEEtran}

%%% Start Preamble
\usepackage[T1]{fontenc}
\usepackage[utf8]{inputenc}
\usepackage[english]{babel}
\AtBeginDocument{%
	\providecommand\BibTeX{{%
			\normalfont B\kern-0.5em{\scshape i\kern-0.25em b}\kern-0.8em\TeX}}}

\usepackage{hyperref}
\usepackage{cite}

\usepackage{amsmath,amssymb,amsfonts} %amsmath already included
\usepackage[noend]{algpseudocode}
\usepackage{algorithmicx, algorithm}
\usepackage{graphicx} %graphicx already included
\usepackage{textcomp}
\usepackage{xcolor}

\usepackage{tikz}
\usetikzlibrary{arrows,decorations.markings, positioning}
\usepackage{subcaption}
\usepackage{tabu}

\hyphenation{con-straints}
%\pagestyle{plain}

%% COMMANDS:
\usepackage{resources/mathcommands}

%% DRAW SCHEDULES:
\usepackage{resources/scheduling}

%% THEOREMS:
% Theorems --
\usepackage{amsthm}

\newtheorem{thm}{Theorem}%[section]
\newtheorem{prop}[thm]{Proposition}
\newtheorem{lem}[thm]{Lemma}
\newtheorem{cor}[thm]{Corollary}
\newtheorem{obs}[thm]{Observation}

\theoremstyle{definition}
\newtheorem{defn}[thm]{Definition}
\newtheorem{notn}[thm]{Notation}

\theoremstyle{remark}
\newtheorem{rmk}[thm]{Remark}
\newtheorem{exmpl}[thm]{Example}
% --


%% WRITING:
\newcommand{\comment}[3]{{\color{#1}#2:#3}\ClassWarning{}{There are comments left in the paper. [Remove them before submission!]}}
\newcommand{\mario}[1]{\comment{red}{mg}{#1}}
\newcommand{\kuan}[1]{\comment{blue}{kh}{#1}}
\newcommand{\jj}[1]{\comment{orange}{jj}{#1}}

\usepackage{lineno}
%\usepackage[switch]{lineno}
\linenumbers
% fix problem with line numbering around align
\makeatletter
\let\LN@align\align
\let\LN@endalign\endalign
\renewcommand{\align}{\linenomath\LN@align}
\renewcommand{\endalign}{\LN@endalign\endlinenomath}
\makeatother


%% PAPER SPECIFIC:


%%% End Preamble




\begin{document}
	%% TITLE:
	\title{Optimal Task Phasing for End-To-End Latency in Semi-Harmonic Task Systems}
	
	%% AUTHORS: Mario, Matthias
	
%	% Final version
%	\author{
%		\IEEEauthorblockN{First Author, Second Author and Third Author}
%		\IEEEauthorblockA{TU Dortmund University\\
%			Email: \{first.author, second.author, third.author\}@tu-dortmund.de}
%		\and
%		\IEEEauthorblockN{Other Author}
%		\IEEEauthorblockA{Other University\\
%			Email: other.author@other-university.de}
%		}
	
%	% Submission version:
	\author{RTAS 2025 Submission \# \textcolor{red}{TBD} \quad\quad\quad Pages: \pageref{last-page}}
	
	\maketitle

	\begin{abstract}
		% - introductory sentence to the topic (optional)
		In the context of automotive systems, the end-to-end latency of a sequence of tasks (a so-called cause-effect chains) is a common metric to ensure correct timing behavior.
		% - describe the issue or main problem (create a need)
		To control the end-to-end latency, proper task configuration is crucial. 
		While the literature considers the configuration of task periods, optimization of task phases to minimize the end-to-end latency is only sparsely discussed. 

		% - one sentence detailing your approach (meet the need)
		In this work, we discuss the configuration of task phases to optimize the end-to-end latency of a cause-effect chain.
		% - list main research results
		To that end, we develop a strategy for semi-harmonic task systems, which are very common in industrial applications.
		% - finish with MAIN MESSAGE of the paper
		We prove that our strategy is optimal in the sense that it minimizes the end-to-end latency.
		Furthermore, our evaluation based on a real-world application shows that optimizing task phases can reduce end-to-end latencies significantly. 
	\end{abstract} 
	
%	\vspace{-.2in}
	
	\section{Introduction}
	\label{sec:introduction}
	% General introduction

	% Topic of this paper
	
	% Detailed topic / tell more details


	State of the art: Release all tasks at the same time


	- Related work: Alessandro ECRTS
	
	
	% Contribution
	\noindent\textbf{Contributions:} % short text
	\begin{itemize}
		\item one
		\item two
		\item three
	\end{itemize}
	
\section{System Model}
\label{sec:system_model}

\section{Motivating Example}

	Show that releasing all jobs at the same time is not optimal, and that you can get better results by manipulating phases.
	
\section{Problem Definition}
\label{sec:problem_def}

	What is the best phasing? Nobody has looked into that yet. 

\section{Optimal Task Phasing for Harmonic Systems}

	Theorem: Phases need to be set at 


	Proposition: How big is the gain that we get wrt synchronous?



\section{Optimal Task Phasing for Semi-Harmonic Systems}

	Look at task set with automotive periods 1,2,5,10,20,50,100,200,1000


	Thm: If all periods divide the largest period, the strategy from previous section still optimal phasing.

	If it does not divide, we need to make space bigger. We do that as follows \dots

	- immer wenn von größter Periode das erste mal zu nicht harmonischer Periode, um die Hälfte schieben 

	Thm: Optimal phasing with this strategy for automotive periods. 

	Proof: 
	- hyperperiod is at most twice as much, therefore at most two different partitioned job chains 
	- Length at least sum of all + phasings from each non harmonic task 
	- Can count number of times jumping from between non harmonic -> loosing at least one half every time 



\section{Evaluation}
\label{sec:evaluation}
	
	Apply our phasing in one or two real applications/ to workload based on real applications and report the gain. 


	- ohne offset
	- random offset
	- unser offset 

	- arbitrary phasing: use Matthias analysis (MRDA + one period)
	
%\vspace{-0.05in}
\section{Conclusion}
\label{sec:conclusion}
	


% bib
\label{last-page}
\clearpage
%\bibliographystyle{abbrv}
%\bibliography{real-time}
	
\end{document}
